\documentclass[oneside]{book}
% Do not indent paragraphs and insert blank space between paragraphs.
\usepackage{parskip}
\usepackage{graphicx}
% For align* environment
\usepackage{amsmath}

\usepackage{hyperref}

\author{
    Sunaina Pai \\
    Susam Pal
}

\title{
    CLRS Notes and Solutions \\
    \large Introduction to Algorithms, 3rd ed.
}
\date{}

\newenvironment{itenv}[2]
               {\begin{trivlist}\item \textbf{#1 #2.} \itshape}
               {\end{trivlist}}
\newenvironment{exercise}[1]{\begin{itenv}{Exercise}{#1}}{\end{itenv}}
\newenvironment{problem}[1]{\begin{itenv}{Problem}{#1}}{\end{itenv}}
\newenvironment{solution}{\textit{Solution.}}{\vspace{12pt}}

\begin{document}
\frontmatter
\maketitle
% URLs
\newcommand{\ccby}[1]
           {\href{http://creativecommons.org/licenses/by/4.0/}{#1}}
\newcommand{\ccbylc}[1]
           {\href{https://creativecommons.org/licenses/by/4.0/legalcode}{#1}}

% Copyright
\copyright{} 2018 Sunaina Pai and Susam Pal

% Logo 
\ccby{\includegraphics[scale=0.75]{ccby.png}}

% License
This document is licensed under the
\ccby{Creative Commons Attribution 4.0 International License}.

% Notice
You are free to share the material in any medium or format and/or adapt
the material for any purpose, even commercially, under the terms of the
Creative Commons Attribution 4.0 International (CC BY 4.0) License.

% Disclaimer
This document is provided \textbf{as-is and as-available,}
\textbf{without representations or warranties of any kind,} whether
express, implied, statutory, or other. See the \ccbylc{CC BY 4.0 Legal
Code} for details.
\tableofcontents
\mainmatter


% I
\part{Foundations}

% 1
\chapter{The Role of Algorithms in Computing}

\section*{Exercises}

% 1.1-1
\begin{exercise}{1.1-1}
Give a real-world example that requires sorting or a real-world example
that requires computing a convex hull.
\end{exercise}

\begin{solution}
A real world example of sorting is sorting files of employees in the
ascending order of employee identification numbers.

A real world example of computing a convex hull is putting a fence
around one's property.
\end{solution}

% 1.1-2
\begin{exercise}{1.1-2}
Other than speed, what other measures of efficiency might one use in a
real-world setting?
\end{exercise}

\begin{solution}
Other than speed, one might use cost, space consumption, manpower, etc. as
measures of efficiency.
\end{solution}

% 1.1-3
\begin{exercise}{1.1-3}
Select a data structure that you have seen previously, and discuss its
strengths and limitations.
\end{exercise}

\begin{solution}
An array.

An array allows constant time lookup of any element by index. However
insertion or deletion of an element at a particular index in the array
requires linear time shifting of other elements at higher indices.
\end{solution}

% 1.1-4
\begin{exercise}{1.1-4}
How are the shortest-path and traveling-salesman problems given above
similar? How are they different?
\end{exercise}

\begin{solution}
The shortest-path and traveling-salesman problems have the following
similarities:
\begin{itemize}
\item They are both graph traversal problems.
\item They both require finding the shortest path in a graph.
\end{itemize}

They have the following differences:
\begin{itemize}
\item The shortest-path problem does not require visiting all the nodes
      in a graph but the traveling-salesman problem requires visiting
      all the nodes in a graph.
\item The starting node and ending node are different in the
      shortest-path problem but they are same in the traveling-salesman
      problem.
\end{itemize}
\end{solution}

% 1.1-5
\begin{exercise}{1.1-5}
Come up with a real-world problem in which only the best solution will
do. Then come up with one in which a solution that is ``approximately''
the best is good enough.
\end{exercise}

\begin{solution}
A real-world problem in which only the best solution will do is to
determine the position of all aircrafts on radar. An error in
determining the presence or location of an aircraft may lead to
accidents.

A problem in which a solution that is approximately the best is good
enough is to determine an estimated time of arrival at a destination
while driving with a navigation system. An error of a few minutes is
usually not disastrous.
\end{solution}

% 1.2-1
\begin{exercise}{1.2-1}
Give an example of an application that requires algorithmic content at
the application level, and discuss the function of the algorithms
involved.
\end{exercise}

\begin{solution}
An example of an application that requires algorithmic content is an
article recommendation system that recommends articles to users that may
be of interest to them. The algorithm needs to determine an article from
the list of unread articles that best matches the user's interest
determined from the reading history of the user.
\end{solution}

% 1.2-2
\begin{exercise}{1.2-2}
Suppose we are comparing implementations of insertion sort and merge
sort on the same machine. For inputs of size \( n \), insertion sort
runs in \( 8n^2 \) steps, while merge sort runs in \(64 n \lg n\) steps.
For which values of \( n \) does insertion sort beat merge sort?
\end{exercise}

\begin{solution}
We want to find integers \( n \) such that \( 8n^2 < 64 n \lg n \). From
this inequality, we get
\begin{align*}
8n^2 < 64 n \lg n & \iff n < 8 \lg n \\
                  & \iff 8 \lg n - n > 0.
\end{align*}

Let \( f(n) = 8 \lg n - n \). To find the critical point of \( f(n) \),
we need to find \( n \) such that \( \frac{d f(n)}{dn} = 0 \). Solving
this, we get
\begin{align*}
\frac{d}{dn} (8 \lg n - n) = 0
& \implies 8 \frac{d}{dn} \left( \frac{\ln n}{\ln 2} - n \right)= 0 \\
& \iff \frac{8}{n \ln 2} - 1 = 0 \\
& \iff n = \frac{8}{\ln 2} \approx 11.54.
\end{align*}

Now we compute the value of \( f(n) \) for some interesting values of
\( n \):
\begin{itemize}
\item f(1) = -1
\item f(2) = 6
\item f(43) = 0.41
\item f(44) = -0.32
\end{itemize}

These values show that \( f(n) > 0 \) when \( 2 \leq n \leq 43 \).

From the fact that \( n = \frac{8}{\ln 2} \approx 11.54 \) is the only
critical point for \( f(n) \) and the above enumerated values, we
conclude that \( f(n) < 0 \) for \( n < 2 \) and \( n > 43 \).

Therefore we conclude that \( f(n) > 0 \) if and only if
\( 2 \leq n \leq 43 \).

We have shown that insertion sort beats merge sort when
\( 2 \leq n \leq 43 \).
\end{solution}
\end{document}
